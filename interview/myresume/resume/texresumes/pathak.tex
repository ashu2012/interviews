\documentclass[letterpaper,11pt]{article}
%-----------------------------------------------------------
%Margin setup

\setlength{\voffset}{0.0in}
\setlength{\paperwidth}{8.5in}
\setlength{\paperheight}{11in}
\setlength{\headheight}{0in}
\setlength{\headsep}{0in}
\setlength{\textheight}{11in}
\setlength{\textheight}{9.5in}
\setlength{\topmargin}{-0.25in}
\setlength{\textwidth}{7in}
\setlength{\topskip}{0in}
\setlength{\oddsidemargin}{-0.25in}
\setlength{\evensidemargin}{-0.25in}
%-----------------------------------------------------------
%\usepackage{fullpage}
\usepackage{shading}
%\textheight=9.0in
\pagestyle{empty}
\raggedbottom
\raggedright
\setlength{\tabcolsep}{0in}

%-----------------------------------------------------------
%Custom commands
\newcommand{\resitem}[1]{\item #1 \vspace{-2pt}}
\newcommand{\resheading}[1]{{\large \parashade[.9]{sharpcorners}{\textbf{#1 \vphantom{p\^{E}}}}}}
\newcommand{\ressubheading}[4]{
\begin{tabular*}{6.5in}{l@{\extracolsep{\fill}}r}
		\textbf{#1} & #2 \\
		\textit{#3} & \textit{#4} \\
\end{tabular*}\vspace{-6pt}}
%-----------------------------------------------------------
\begin{document}
\begin{tabular*}{7in}{l@{\extracolsep{\fill}}r}
\textbf{\Large Ashutosh Pathak}  & +91-9415540881\\
E-306,Hall-8 &  apathak@iitk.ac.in \\
Indian Institute of technology Kanpur  & http://home.iitk.ac.in/\~apathak\\
\end{tabular*}
\\
\vspace{0.1in}
\resheading{Education}
\begin{tabular}{|c|c|c|}
\hline
\bf Degree/Certificate&\bf Institute/School,City&\bf CGPA/\%\\
\hline
Bachelor \& Masters of Technology (Computer Science)&Indian Institute of Technology, Kanpur&8.0/10.0\\
\hline
Bachelor  of Technology (Computer Science)&Indian Institute of Technology, Kanpur&6.1/10.0\\
\hline
Masters of Technology (Computer Science)&Indian Institute of Technology, Kanpur&8.0/10.0\\
\hline
Current CGPA &Indian Institute of Technology, Kanpur&8.0/10.0\\
\hline
All India Senior School Certificate Examination-XII&Senior Seconder School Sector -X ,Bhilai &73\%\\
\hline
All India Secondary School Examination-X& Kendriya Vidyalaya Sarni,MP&83.6\%\\
\hline

\end{tabular}
\resheading{Projects}
\begin{itemize}
\item
\ressubheading{Summer internship}{Paris, France}{National Des telecomunication University }{may. 2007 - July. 2007}
		\begin{itemize}
		\resitem{basic concepts of watermarking}
		\resitem{implement watermarking on the video }
		\resitem{research paper in the conferance in callifornia }
		\end{itemize}
		
\item
	\ressubheading{Extension of Nachos}{}{Prof. Manindra Agarwal, Department of Computer Science, IIT Kanpur}{}	
		\begin{itemize}
		\resitem{Worked in group of three as a part of Operating Systems Course}
		\resitem{implemented some features like System Calls, Scheduling Algorithms, Multiprogramming and Virtual Memory.}
		\end{itemize}
	
\item
	\ressubheading{UDP-Chat system}{}{Guide: Dr. Bhaskar Raman, CSE Department, IIT Kanpur}{(Fall 2006)}	
		\begin{itemize}
		\resitem{Developed a chat system with functionalities of conferencing and file sending.}
		\resitem{Designed a protocol to implement reliability, flow control and congestion control over
transport layer in User Datagram Protocol (UDP) using socket programming.}
		\end{itemize}
	
\item
	\ressubheading{Implementation of SDLX Processor}{}{Dr. Rajat Moona, CSE Department, IIT Kanpur}{Spring 2005}	
		\begin{itemize}
		\resitem{implementation of SDLX processor having Instruction-fetch \& Register-decode unit,ALU, Register file and MEM unit with delayed branch semantics}
		\resitem{Work was done on Spartan-3(FPGA board)in Verilog.}
		\end{itemize}

\item
	\ressubheading{Oberon Compiler}{}{Dr. Sanjeev Kumar Agarwal, CSE Department, IIT Kanpur}{Spring 2007}	
		\begin{itemize}
		\resitem{As a part of course curriculum of Compiler Design, I with two of my colleges, are
developed a compiler for Oberon language in java with almost all  features}
		\resitem{We have completed lexical , Syntax Analyzer and  Semantic analyzer.
We have also proposed to implement Code Optimizer with Intermediate Language
Generation and Code Generation.}
		\end{itemize}		
		
		\item
	\ressubheading{PHP based webchat}{}{Prof. T. V. Prabhakar, CSE Department, IIT Kanpur}{Fall 2007}	
		\begin{itemize}
		\resitem{implemented with the software engineering concepts using UML diagrams.}
		\resitem{It is modular and follow the MVC design pattern}
		\end{itemize}	
		
		\item
	\ressubheading{Peer to Peer chat}{}{Prof. T. V. Prabhakar, CSE Department, IIT Kanpur}{Spring 2008}	
		\begin{itemize}
		\resitem{it is decentralised p2p chat using java's JXTA technology.}
		\resitem{Bare minimum chat with peer table and discovery .}
		\end{itemize}	
		
		\item
	\ressubheading{Solver for Very Large System of Equations}{}{Prof. Phalguni Gupta, Department of Computer Science, IIT Kanpur}{Spring 2008}	
		\begin{itemize}
		\resitem{in a group of two, successfully developed a solver for a large system of equations of order 1000x1000 where each of the coefficients being of the order of 500 digits in a finite field using LU decompostion }
		\resitem{It is for clusters and mesh connected computers, with GF2 field and both real numbers}
		\end{itemize}				
		
			\item
	\ressubheading{Ofline biometric signature Verification}{}{Prof. Phalguni Gupta, Department of Computer Science, IIT Kanpur}{Fall 2008}	
		\begin{itemize}
		\resitem{in a group of two, successfully developed a signature verification system using SVM which can used for scanned images of signature. }
		\resitem{}
		\end{itemize}				
		
		
			\item
	\ressubheading{Similarity Search in time series database}{}{Prof. Arnab Bhatacharya, Department of Computer Science, IIT Kanpur}{Fall 2008}	
		\begin{itemize}
		\resitem{I have made this as single person , the aim of the course project is to study search methods for long time series databases.  }
		\resitem{}
		\end{itemize}		
		
			\item
	\ressubheading{motion detector}{}{Prof. R. K. Rathore, Department of Methamatics , IIT Kanpur}{spring 2006}	
		\begin{itemize}
		\resitem{in a group of two, we have made a software which can detect moviing object in video taken in still camera position by background subtraction algorithm  . }
		\resitem{It was part of the course project of "Digital Image processing"}
		\end{itemize}			
		
			\item
		\ressubheading{Prefetching \& Caching protocols for Wide Area Wireless data dissemination}{}{R. K. Ghosh, CSE Department, IIT Kanpur}{Fall Of 2007}	
		\begin{itemize}
		\resitem{Problem is in prediction of future data access and it can be severe if the Mobile user is going from one
hot-spot to another.}
		\resitem{Idea is to potentially predict what user might need in near future and prefetch and cache the data}
		\resitem{low bandwidth network that can result in really high latency. So, correct prediction of future data access is highly neccessary}
		\resitem{we use user�s speed, location, route, history to predict future data access.}
		\end{itemize}	
		
			\item
		\ressubheading{Concurent vi}{}{R. K. Ghosh, CSE Department, IIT Kanpur}{Spring 2008}	
		\begin{itemize}
		\resitem{to make vi a concurrent editor of linux}
		\resitem{to implement locks in the data section}
		\resitem{code was submitted on sourge forg and get much appreciation from opensource community as it was totally new feature in the vi}
		\end{itemize}	
		
		
			\item
			\ressubheading{Medical Records}{}{Sumit Ganguly,, CSE Department, IIT Kanpur}{Fall 2006}	
		\begin{itemize}
		\resitem{it was part of the database course and was developed using LAMP architecture.}
		\resitem{show statics of which area of india is affected by which disease in which year}
		\resitem{data mining over medical data}
		\end{itemize}	
		
		\item
			\ressubheading{network Analyser}{}{Sumit Ganguly,, CSE Department, IIT Kanpur}{Fall 2007}	
		\begin{itemize}
		\resitem{It was part of the datastreaming course and developed using pcap library ,in linux}
		\resitem{it is based on randomised fast map,count min ,count skectches. }
		\resitem{it can during run time shows the statics of the packet flow and bandwith utilisation by different ips.}
		\end{itemize}	
		
		
			\item
			\ressubheading{dual Stepler}{}{Amit Ray, Dept. Of Huminty \& Social Science,, IIT Kanpur}{Spring 2008}	
		\begin{itemize}
		\resitem{to make single stepler able to use two kind of pins }
		\resitem{It is a project in which we have to design inovative things , we secured A grade in this.}
		\end{itemize}	
		
\end{itemize}


\resheading{Awards}
\begin{description}
\item[]
Securred All India Rank 416  in IIT-JEE exam in 2004 in general category.
\item[]
Secured All India Rank 1200 in the AIEEE 2004 with state rank 12.
 \item[]
 Cleared the first phase and secured the Top 1\% in National Physics Olympiad .\\
 NSEP stands for National Standard Examination in Physics. It is conducted by Indian Association of Physics Teachers (IAPT).
 About 200 top students in NSEP are selected to appear for the Stage II examination, i.e. INPhO. These students get a certificate of merit each.




\end{description}



\resheading{Research \& Technical Capability}
\begin{description}
\item[]
\begin{itemize}
	\item I worked on watermarking videos using dct transformation during Internship
	Achievemnt:- I sugest some improvement and new method for the same which result in
	the paper published in the "Electronic Imaging 2008".I received the appreciation and invitation for the conferance in callifornia , from the professor Mihai mitrea ,whose email* is attached herewith.
\end{itemize}

\item[]
\begin{itemize}
	\item
	Currently I am doing my masters thesis under professor Phalguni Gupta, The title of my thesis is "Onlinre signature verification "  , which is MHRD sponsered project.
\end{itemize}

\item[]
\begin{itemize}
	\item I spent the summer vacation of 2006 after my second year, working at SIMORTEL,in campus ,a SIDBI incubated company. I worked on a GPS based software to track the buses in city which is more likely to be used in future years, written in C.
\end{itemize}
\end{description}




\resheading{Positions of Responsibility}
\begin{description}
\item[]
\begin{itemize}
	\item Elected as techncal assistance(TA) in cs220(Computer Organisation) course.helped the course students in implemntationof ALU , Regfile, Ram , CPU in FPGA Board.
\end{itemize}
\item[]
\begin{itemize}
	\item Mentor the winter project under the sunclub,lead the group of 3 people and helped them in completing the project.
\end{itemize}
\end{description}






\resheading{Skills}

\begin{description}
\item[Languages:]
C/C++, \LaTeX, Java, C,C++,Java,C\#,Php,ruby,shell screpting,python,UML, Verilog, IA32 Assembly.
\item[Operating Systems:]
Linux (Redhat , Ubuntu ,Fedora), Solaris, Windows 95/98/NT/2000/XP
\item[Applications and IDE:]
Mathematica, MatLab,Visual Studio,Eclipse, Netbeans,GCC debugur, OpenOffice, MS Office XP ,Ruby on Rails, LAMP, Apache Web Server, Mysql, Oracle.
\item[Lab Skils and Experiance:]
FPGA Board , Mechnical Lab , Electronics Lab.
\item[Miscellaneous:]
application of software engineering concepts, strong verbal and written communication skills, excellent troubleshooting and debugging skills, exceptional problem solving skills, good team spirit.
\end{description}

\resheading{Interests And Extra Corecullar}
\begin{description}
\item[Academic:]syatems, os, network, databases ,  new web based technologies , microcontrollers.
\item[Sports:]  swimming,I was in Institute swimming team during 1st year at IITK.
\item[Computers:] Currently maintain two official ubuntu Linux packages, Mozilla beta tester, enjoy using and learning Linux systems .
\item[Musical:] Playing synthesizer
\item[technology:] reading latest technology news using Zdnet. C\# coders.
\item[Other:] Reading novels
\end{description}

\resheading{Courses} 
\begin{description}
\item[Relevent:]
1.System related:-\\
 Distributed Systems , Operating Systems  ,  Computer Organisation  ,Fundamental of electronics\\
 , Principles Of DataBase Systems , Compiler Design\\
 2.Theoritical:-\\
 Advance Algorithms ,  , Data Structures  , Parallel and semi Numerical Algorithn \\
 Theory of Computation ,  ,Numerical Methods In Engineering, Linear Algebra\\
 Discreet Mathematics \\
3. Network related:-\\
 Mobile Computing ,  Computer Networks ,  Data Streaming  \\
4. Software :-\\
 Principles Of Programming Languages ,Software Engineering ,Internet technologies ,\\ Programming Tools \&Techniques,Fundametals Of Computing \\
 

 
 \item[other courses:]
 rest are from grad sheet.
 \end{description}
 
 
 \resheading{Attachment} 
\begin{description}
 \item[*email:]
Dear Dr. Rolin,
Dear Professor Pr�teux,

Please, find attached some good news: our paper including some of Ashutosh
results has been accepted for Electronic Imaging 2008.
Of course, this means for Ashutosh some more experiments to do but I am
quite confident he will confirm the very good impression he gave us.
With my highest consideration,
Mihai 


 \end{description}


\end{document}