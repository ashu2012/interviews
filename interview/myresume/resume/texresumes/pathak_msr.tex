\documentclass[margin]{res}
%-----------------------------------------------------------
%Margin setup

\setlength{\voffset}{0.0in}
\pdfpagewidth 8.5in
\pdfpageheight 11in
%\setlength{\paperwidth}{8.5in}
%\setlength{\paperheight}{11in}
\setlength\topmargin{-0.0in}
\setlength{\headheight}{0in}
\setlength{\headsep}{0in}
\setlength{\textheight}{11.0in}
\setlength{\textwidth}{7in}
\setlength{\topskip}{0in}
%\setlength{\footskip){0in}
\setlength{\oddsidemargin}{-1.5in}
\setlength{\evensidemargin}{-1.25in}
%-----------------------------------------------------------
%\usepackage{fullpage}
\usepackage{shading}
%\textheight=9.0in
%\pagestyle{empty}
\raggedbottom
\raggedright
\setlength{\tabcolsep}{0in}

%-----------------------------------------------------------
%Custom commands
\newcommand{\resitem}[1]{\item #1 \vspace{-2pt}}
\newcommand{\resheading}[1]{{\large \parashade[.9]{sharpcorners}{\textbf{#1 \vphantom{p\^{E}}}}}}
\newcommand{\ressubheading}[4]{
\begin{tabular*}{6.5in}{l@{\extracolsep{\fill}}r}
		\textbf{#1} & #2 \\
		\textit{#3} & \textit{#4} \\
\end{tabular*}\vspace{-6pt}}
%-----------------------------------------------------------
\begin{document}
\begin{tabular*}{7in}{l@{\extracolsep{\fill}}r}
\textbf{\Large Ashutosh Pathak} \\ Dual Degree  , Computer Science Engineering & +91-9415540881\\
E-306,Hall-8 &  apathak@iitk.ac.in \\
Indian Institute of technology Kanpur  & http://home.iitk.ac.in/\~{}apathak\\
\end{tabular*}
\\
\vspace{-0.0in}
\resheading{Education}
\vspace{-0.3in}
\begin{tabular}{|c|c|c|}
\hline
\bf Degree/Certificate&\bf Institute/School,City&\bf CGPA/\%\\
\hline
 CGPA &Indian Institute of Technology, Kanpur&8.0/10.0\\
\hline
AISSCE-XII,CBSE&Senior Secondary School Sector -X ,Bhilai &73\%\\
\hline
AISSE-X,CBSE , school toper& Kendriya Vidyalaya Sarni,MP&83.6\%\\
\hline

\end{tabular}

\vspace{-0.1in}

\resheading{Awards}
\vspace{-0.2in}
\begin{description}
\item[]
\begin{itemize}
\item
Secured All India Rank \textbf{416 } in \textbf{IIT-JEE} exam in 2004 in general category.
\end{itemize}

\item[]
\begin{itemize}
 \item Secured All India Rank 1200 in the AIEEE 2004 with state rank 12.
 \end{itemize}
 
 \item[]
 \begin{itemize}
 \item Cleared the first phase and secured the \textbf{Top 1\% in National Physics Olympiad }.\\
 
\end{itemize}

\item[]
\begin{itemize}
\item Secured All India Rank 2 in the ICFAI engineering exam 2004 .
\end{itemize}

\end{description}


\vspace{-0.2in}
\resheading{Research \& Technical Capability}
\vspace{-0.2in}
\begin{description}
\item[]
%\ressubheading{}
\begin{itemize}
	\item \textbf {International Exposure \& Research Paper :-}\\
	 \textbf{INSTITUTE NATIONAL DES TELECOMMUNICATIONS (INT),Paris, France }.\\
	During \textbf{summer internship (may - July 2007)}, my duty was to work on robust Watermarking  Techniques on videos \& Study the different kind of attacks to compromise the watermarking .Work involved Signal theory ,DCT and Wavelet Transform, X.264 codecs, and fast processing of video frames .\textbf{Achievement:-} My suggested choice of  coefficients(pixels) for watermarking \&  better data Structures result in slight improvement  . Finally the paper was  published in the \textbf{"Electronic Imaging 2008"}  with the help of my supervisor Dr Mihai Metria .I received the appreciation and invitation for the conference in california.
\end{itemize}

\item[]
\begin{itemize}
	\item \textbf{Thesis:- }
	"Biometric Online signature verification using Digital Signature Pad ".  Supervisor :- Dr Phalguni Gupta .
This is a MHRD sponsored project.
\end{itemize}

\item[]
\begin{itemize}
	\item \textbf{Work Experience :-}
	During  summer vacation of $2^{nd}$ year,I got the opportunity to work at \textbf{SIMORTEL,  company}. My duty was to work on a GPS based software to track the buses in city which is more likely to be used in near future in India. 2000 lines of Code was written in C.Project involved sql quarries , Dictionary data structures , network sockets and Multi threading .
\end{itemize}
\end{description}


\vspace{-0.2in}

\resheading{Positions of Responsibility}
\vspace{-0.2in}
\begin{description}
\item[]
\begin{itemize}
	\item  \textbf{Teaching Assistant(TA),Computer Organization course ,Aug 2008-ongoing)}\\
	Responsibilities included advising weak students, designing assignments  and  helping the course students in implementation of ALU , Regfile, Ram , CPU in Spartan series 3 FPGA Board in Hardware lab using xilinix IDE and verilog language.
\end{itemize}
\item[]
\begin{itemize}
	\item \textbf{Mentorship:-}I Mentored the winter project under the \textbf{ Sun Club} and  lead the group of 3 people and helped them in completing the project.Project involved the development of online room booking system using JSP and Tomcat with the help of netbeans IDE.
\end{itemize}
\end{description}

\vspace{-0.2in}
\resheading{Courses} 
\vspace{-0.2in}
\begin{description}
\item Relevent:\\
\begin{itemize}
\item
System related:-\\
 Distributed Systems , Operating Systems, Computer Organization ,Fundamental of electronics,
 Principles Of Database Systems , Compiler Design.\\
 \item
 Theoretical:-\\
 Advance Algorithms ,Data Structures ,Parallel and semi Numerical Algorithm ,\\
 Theory of Computation ,Numerical Methods In Engineering, Linear Algebra, Discreet Mathematics .\\
\item
 Network related:-\\
 Mobile Computing ,Computer Networks ,Data Streaming . \\
\item
 Software :-\\
 Principles Of Programming Languages ,Software Engineering ,Internet technologies ,\\
  Programming Tools \&Techniques, Fundamentals Of Computing(JAVA) .\\
 \end{itemize}	

 
 \item[other courses:]
 rest are from grade sheet.
 \end{description}
 


\newpage

\vspace{-0.2in}
\resheading{Major Projects in Chronological Order }
\vspace{-0.2in}
\begin{itemize}
	%%%%%%%%%%%%%%%	
		\item \textbf{Similarity Search in Time Series Database , Fall 2008}
	%\ressubheading{Similarity Search in time series database}{Fall 2008}	
	\vspace{-0.1in}
		\begin{itemize}
		\resitem{Aim was to implement the technique for fast retrieval and search similar time series  in large video \& audio database .  Dynamic Time Warping(DTW ) method is studied for searching  KNN($K^{th }$ nearest neighbor) quarries .}
		
		\end{itemize}		
		
		
%%%%%%%%%%%%%%%%%%
\vspace{-0.1in}
			\item \textbf{User level Thread Library with mutual exclusion support , Spring 2008}	
		%\ressubheading{Concurent vi}{Spring 2008}
		\vspace{-0.1in}	
		\begin{itemize}
		\resitem{
		Implemented user level thread library taking care of issues like scheduling, starvation and fairness using  Linux system calls provided in ucontext.h, signal.h, sched.h, \& setjmp.h like makecontext , swapcontext etc.
Studied various distributed mutual exclusion algorithms (non-token and token based) and measured throughput
and synchronization delays for  them .
		 }
		\end{itemize}	
%%%%%%%%%%%%%%%%%%%%%%		
		\vspace{-0.1in}
			\item \textbf{Concurrent Vi , Spring 2008}	
		%\ressubheading{Concurent vi}{Spring 2008}
		\vspace{-0.1in}	
		\begin{itemize}
		\resitem{Aim was to make \textbf{Vi} a concurrent editor of Linux able to edit same file by more than one process.we implemented user defined locks in the data section for concurrency.Code was submitted on sourceforg.net and get much appreciation from open source community as it was totally new feature in the Vi  .Project involved going into the vim source code of more that 6000 lines. }
		\end{itemize}	
%%%%%%%%%%%%%%%%%%%%%%
\vspace{-0.1in}		
			\item \textbf{Parallel Solver for Very Large System of Equations , Spring 2008}
	%\ressubheading{Solver for Very Large System of Equations}{Spring 2008}	
		\vspace{-0.1in}
		\begin{itemize}
		\resitem{ Successfully developed a solver for a large system of equations of order 50000x50000 using MPI and openMP .It is for clusters and mesh connected computers, with inputs in , GF2  and  real field . }
		
		\end{itemize}			
%%%%%%%%%%%%%%%%%%%%%%%%%%%	
	\vspace{-0.1in}
	\item \textbf{Oberon Compiler , Spring 2007}
	%\ressubheading{Oberon Compiler}{Spring 2007}	
		\vspace{-0.1in}
		\begin{itemize}
		\resitem{ Developed a bare minimal compiler for Oberon language in java with almost all  features having  lexical ,Syntax Analyzer and  Semantic analyzer phase .It can compile programs involving recursion and simple case statements.4000 lines of code java}
		\end{itemize}		
		
%%%%%%%%%%%%%%%%%%%%%%%%%			
		\vspace{-0.1in}
			\item \textbf{Dual Stapler , Spring 2007}	
		%	\ressubheading{Dual Stepler}{Spring 2007}	
		\vspace{-0.1in}
		\begin{itemize}
		\resitem{As a part of ART course project we made a single stapler able to use two kind of pins . we  designed a innovative multipurpose stapler, we secured \textbf{A grade} in this.}
		\end{itemize}	
%%%%%%%%%%%%%%%%%%%%%%%%%%%		
		\vspace{-0.1in}
			\item \textbf{Medical Records Spring , 2007}
		%	\ressubheading	
		\vspace{-0.1in}
		\begin{itemize}
		\resitem{ Developed a web server and database driven management system  using LAMP architecture.It can show statistics of population, diseases, region and  year . Work involved Apache  and mysql database .}
	
		\end{itemize}	
%%%%%%%%%%%%%%%%%%%%%				
		\vspace{-0.1in}
\item \textbf{Extension of Nachos , Fall 2006}
%	\ressubheading{Extension of Nachos}{Fall 2006}	
		\vspace{-0.1in}
		\begin{itemize}
		\resitem{Implemented some features like System Calls, Scheduling Algorithms, Multiprogramming and Virtual Memory on Nachos  operating system in C++.}
		\end{itemize}
	
	
%%%%%%%%%%%%%%%%%%%%%%
\vspace{-0.1in}	
\item \textbf{UDP-Chat System ,Fall 2006}	
	%\ressubheading{UDP-Chat system}{Fall 2006}	
		\vspace{-0.1in}
		\begin{itemize}
		%{UDP-Chat system}{Fall 2006}	
		\resitem{We Developed a chat system with functionalities of conferencing and file sending .We have implemented reliability, flow control and congestion control over transport layer in User Datagram Protocol (UDP) using socket programming.}
		\end{itemize}
		
%%%%%%%%%%%%%%%%%%%%%%%%%%%%%%%%%%%%%	
\vspace{-0.1in}
\item	\textbf{Implementation of Mips Processor Fall 2005}
		\begin{itemize}
		\vspace{-0.1in}
		\resitem{Implementation of MIPS processor having Instruction-fetch \& Register-decode unit,ALU, Register file and MEM unit using SRam and Block Ram .Work was done on Spartan-3(FPGA board)in Verilog with help of Xilinix IDE.}
	
		\end{itemize}

		
\end{itemize}





\vspace{-0.2in}
\resheading{Skills}
\vspace{-0.2in}
\begin{description}
\item[Languages:]
C/C++, \LaTeX, Java,C\#,Php,ruby,bash shell scripting,UML, Verilog, IA32 Linux Assembly ,xml, Sql.
\item[Operating Systems:]
Linux (Redhat , Ubuntu ,Fedora), Solaris, Windows 95/98/NT/2000/XP
\item[Applications and IDE:]
Mathematica, MatLab,Visual Studio,Eclipse, Netbeans,GCC debugger, OpenOffice, MS Office XP ,Ruby on Rails, LAMP, Apache Web Server, Mysql, Oracle.
\item[Lab Skills and Experience:]
FPGA Board , Mechanical Lab , Electronics Lab.
\item[Miscellaneous:]
Application of software engineering concepts, strong verbal and written communication skills, excellent troubleshooting and debugging skills, exceptional problem solving skills, good team spirit.
\end{description}

\vspace{-0.2in}
\resheading{Interests And Extra Curricular}
\vspace{-0.2in}
\begin{description}
\item[Academic:]systems, os, network, databases ,  new web based technologies , microcontrollers ,Finance .
\item[Sports:]  football ,swimming.\textbf{I was in Institute swimming team during 1st year at IITK}.
\item[Computers:]  Mozilla beta tester, enjoy using and learning Linux systems .
\item[technology:] reading latest technology news using Zdnet ,Code project.
\item[Other:] Reading novels,Stocks Rate  .
\end{description}
 
 


 

\end{document}