\documentclass[letterpaper,11pt]{article}
%-----------------------------------------------------------
%Margin setup

\setlength{\voffset}{0.0in}
\setlength{\paperwidth}{8.5in}
\setlength{\paperheight}{11in}
\setlength{\headheight}{0in}
\setlength{\headsep}{0in}
\setlength{\textheight}{11in}
\setlength{\textheight}{10.0in}
\setlength{\topmargin}{-0.25in}
\setlength{\textwidth}{7in}
\setlength{\topskip}{0in}
\setlength{\oddsidemargin}{-0.25in}
\setlength{\evensidemargin}{-0.25in}
%-----------------------------------------------------------
%\usepackage{fullpage}
\usepackage{shading}
%\textheight=9.0in
\pagestyle{empty}
\raggedbottom
\raggedright
\setlength{\tabcolsep}{0in}

%-----------------------------------------------------------
%Custom commands
\newcommand{\resitem}[1]{\item #1 \vspace{-2pt}}
\newcommand{\resheading}[1]{{\large \parashade[.9]{sharpcorners}{\textbf{#1 \vphantom{p\^{E}}}}}}
\newcommand{\ressubheading}[4]{
\begin{tabular*}{6.5in}{l@{\extracolsep{\fill}}r}
		\textbf{#1} & #2 \\
		\textit{#3} & \textit{#4} \\
\end{tabular*}\vspace{-6pt}}
%-----------------------------------------------------------
\begin{document}
\begin{tabular*}{7in}{l@{\extracolsep{\fill}}r}
\textbf{\Large Ashutosh Pathak}  & +91-9415540881\\
E-306,Hall-8 &  apathak@iitk.ac.in \\
Indian Institute of technology Kanpur  & http://home.iitk.ac.in/\~{}apathak\\
\end{tabular*}
\\
\vspace{0.1in}
\resheading{Education}
\begin{tabular}{|c|c|c|}
\hline
\bf Degree/Certificate&\bf Institute/School,City&\bf CGPA/\%\\
\hline
Current CGPA &Indian Institute of Technology, Kanpur&8.0/10.0\\
\hline
AISSCE-XII,CBSE&Senior Seconder School Sector -X ,Bhilai &73\%\\
\hline
AISSE-X,CBSE , school toper& Kendriya Vidyalaya Sarni,MP&83.6\%\\
\hline

\end{tabular}



\resheading{Awards}
\begin{description}
\item[]
\begin{itemize}
\item
Securred All India Rank 416  in IIT-JEE exam in 2004 in general category.
\end{itemize}

\item[]
\begin{itemize}
 \item Secured All India Rank 1200 in the AIEEE 2004 with state rank 12.
 \end{itemize}
 
 \item[]
 \begin{itemize}
 \item Cleared the first phase and secured the Top 1\% in National Physics Olympiad .\\
 
\end{itemize}

\item[]
\begin{itemize}
\item Secured All India Rank 2 in the ICFAI engineering exam 2004 .
\end{itemize}

\end{description}



\resheading{Research \& Technical Capability}
\begin{description}
\item[]
\begin{itemize}
	\item I worked on watermarking videos using dct transformation during Internship
\textbf{Achievemnt:-} My sugested improvement and  speed up for the same , result in
	the paper published in the "Electronic Imaging 2008".I received the appreciation and invitation for the conferance in callifornia , from the Intern Professor .
\end{itemize}

\item[]
\begin{itemize}
	\item
	Currently I am doing my masters thesis under professor Dr Phalguni Gupta , The title of my thesis is "Biometric Online signature verification using digital Pad "  , which is MHRD sponsered project.
\end{itemize}

\item[]
\begin{itemize}
	\item I spent the summer vacation of $2^{nd}$ year,at SIMORTEL, a SIDBI incubated company. I worked on a GPS based software to track the buses in city which is more likely to be used in future years in India, written in C.
\end{itemize}
\end{description}




\resheading{Positions of Responsibility}
\begin{description}
\item[]
\begin{itemize}
	\item Elected as technical assistance(TA) in cs220(Computer Organisation) course.Our duty was to help the course students in implemntationof ALU , Regfile, Ram , CPU in Spartan series 3 FPGA Board.
\end{itemize}
\item[]
\begin{itemize}
	\item I Mentored the winter project under the sunclub and  lead the group of 3 people and helped them in completing the project.
\end{itemize}
\end{description}

\newpage

\resheading{Projects}
\begin{itemize}
\item
\ressubheading{Summer internship}{Paris, France}{INSTlTUTE NATIONAL DES TELECOMMUNICATIONS (INT) }{may. 2007 - July. 2007}
		\begin{itemize}
		\resitem{Implemention of  watermarking on the videos. }
		\end{itemize}
		
\item
	\ressubheading{Extension of Nachos}{}{Dr. Manindra Agarwal, CSE Department, IIT Kanpur}{Fall 2006}	
		\begin{itemize}
		\resitem{Implemented some features like System Calls, Scheduling Algorithms, Multiprogramming and Virtual Memory on Nachos  operating system.}
		\end{itemize}
	
\item
	\ressubheading{UDP-Chat system}{}{ Dr. Bhaskar Raman, CSE Department, IIT Kanpur}{(Fall 2006)}	
		\begin{itemize}
		\resitem{We Developed a chat system with functionalities of conferencing and file sending .We have implemented reliability, flow control and congestion control over transport layer in User Datagram Protocol (UDP) using socket programming.}
		\end{itemize}
	
\item
	\ressubheading{Implementation of SDLX Processor}{}{Dr. Rajat Moona, CSE Department, IIT Kanpur}{Spring 2005}	
		\begin{itemize}
		\resitem{Implementation of SDLX processor having Instruction-fetch \& Register-decode unit,ALU, Register file and MEM unit .Work was done on Spartan-3(FPGA board)in Verilog.}
	
		\end{itemize}

\item
	\ressubheading{Oberon Compiler}{}{Dr. Sanjeev Kumar Agarwal, CSE Department, IIT Kanpur}{Spring 2007}	
		\begin{itemize}
		\resitem{ I with two of my colleges,developed a bare minimal compiler for Oberon language in java with almost all  features having  lexical ,Syntax Analyzer and  Semantic analyzer phase .}
		\end{itemize}		
		
	
		\item
	\ressubheading{Solver for Very Large System of Equations}{}{Dr. Phalguni Gupta,  CSE Department, IIT Kanpur}{Spring 2008}	
		\begin{itemize}
		\resitem{In a group of two,we  successfully developed a solver for a large system of equations of order 1000x1000 .It is for clusters and mesh connected computers, with GF2 field and  real field }
		
		\end{itemize}				
		
		
			\item	
	\ressubheading{Offline biometric signature Verification}{}{Dr. Phalguni Gupta,  CSE Department, IIT Kanpur}{Fall 2008}	
		\begin{itemize}
		\resitem{We developed a signature verification system using SVM(Support Vector Machines ) which can be used for scanned images of signature. }
		\end{itemize}				
		
		
			\item
	\ressubheading{Similarity Search in time series database}{}{Dr. Arnab Bhatacharya, Department of Computer Science, IIT Kanpur}{Fall 2008}	
		\begin{itemize}
		\resitem{I have implemented Dynamic Time Warping(DTW ) method to  search KNN querries in long time series databases like videos , audios etc.  }
		
		\end{itemize}		
		
		
			\item
		\ressubheading{Concurent vi}{}{Dr. R. K. Ghosh, CSE Department, IIT Kanpur}{Spring 2008}	
		\begin{itemize}
		\resitem{To make \textbf{Vi} a concurrent editor of linux .we implemented user defined locks in the data section .code was submitted on sourceforg.net and get much appreciation from opensource community as it was totally new feature in the vi}
		\end{itemize}	
		
		
			\item
			\ressubheading{Medical Records}{}{Dr. Sumit Ganguly,, CSE Department, IIT Kanpur}{Fall 2006}	
		\begin{itemize}
		\resitem{it was developed using LAMP architecture.It can show statistics of population, diseases, region and  year.}
	
		\end{itemize}	
		

		
			\item
			\ressubheading{Dual Stepler}{}{Dr. Amit Ray, Dept. Of Huminty \& Social Science,, IIT Kanpur}{Spring 2007}	
		\begin{itemize}
		\resitem{To make single stepler able to use two kind of pins . we have  designed a inovative multipurpose stepler, we secured A grade in this.}
		\end{itemize}	
		
\end{itemize}



\resheading{Courses} 
\begin{description}
\item Relevent:\\
\begin{itemize}
\item
System related:-\\
 Distributed Systems , Operating Systems  ,  Computer Organisation  ,Fundamental of electronics,
 Principles Of DataBase Systems , Compiler Design.\\
 \item
 Theoritical:-\\
 Advance Algorithms ,  , Data Structures  , Parallel and semi Numerical Algorithn ,\\
 Theory of Computation ,  ,Numerical Methods In Engineering, Linear Algebra,\\
 Discreet Mathematics .\\
\item
 Network related:-\\
 Mobile Computing ,  Computer Networks ,  Data Streaming . \\
\item
 Software :-\\
 Principles Of Programming Languages ,Software Engineering ,Internet technologies ,\\
  Programming Tools \&Techniques,Fundametals Of Computing .\\
 \end{itemize}	

 
 \item[other courses:]
 rest are from grade sheet.
 \end{description}
 





\resheading{Skills}

\begin{description}
\item[Languages:]
C/C++, \LaTeX, Java,C\#,Php,ruby,shell screpting,UML, Verilog, IA32 Linux Assembly.
\item[Operating Systems:]
Linux (Redhat , Ubuntu ,Fedora), Solaris, Windows 95/98/NT/2000/XP
\item[Applications and IDE:]
Mathematica, MatLab,Visual Studio,Eclipse, Netbeans,GCC debugur, OpenOffice, MS Office XP ,Ruby on Rails, LAMP, Apache Web Server, Mysql, Oracle.
\item[Lab Skils and Experiance:]
FPGA Board , Mechnical Lab , Electronics Lab.
\item[Miscellaneous:]
Application of software engineering concepts, strong verbal and written communication skills, excellent troubleshooting and debugging skills, exceptional problem solving skills, good team spirit.
\end{description}

\resheading{Interests And Extra Corecullar}
\begin{description}
\item[Academic:]syatems, os, network, databases ,  new web based technologies , microcontrollers.
\item[Sports:]  swimming,I was in Institute swimming team during 1st year at IITK.
\item[Computers:]  Mozilla beta tester, enjoy using and learning Linux systems .
\item[technology:] reading latest technology news using Zdnet ,Code project.
\item[Other:] Reading novels
\end{description}
 
 


 

\end{document}